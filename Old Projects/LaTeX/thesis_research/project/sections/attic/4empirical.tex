\documentclass[../main.tex]{subfiles}

\doublespacing

\begin{document}

\subsection{Estimation Strategy}
In order to evaluate the effect of honey
production on food insecurity, we estimate
the following regressions. In all of the regressions, 
the outcome of interest is a dummy variable $y_{i,m}$ that indicates
the presence of reported food insecurity in a given month.
We choose a linear probability model for the ease of interpretation
of the coefficients and because of the instrumental variable estimation
strategy we will propose in the section.

Equation \ref{eq:month} estimates
the baseline monthly variation in the outcome of interest,
Equation \ref{eq:month_fe} incorporates
individual fixed effects and equation \ref{eq:month_sd} incorporates
a vector of demographic characteristics (gender, age, and education level).

We are interested in any differences between cooperative members and non-members.
Because cooperative membership is a choice variable and potentially correlated
with observable demographic characteristics and unobservables, we estimate
equation \ref{eq:month_sd} on subgroups of members and non-members separately.
In all cases, estimates of $\delta_m$ capture monthly differences in the
probability of being food insecure.
\begin{align}
\label{eq:month}
 y_{i,m} &= \sum_{m=2}^{12} \delta_{m1} \text{month}_{m} + \epsilon_{i,m}  \\
 \label{eq:month_fe}
 y_{i,m} &= \sum_{m=2}^{12} \delta_{m2} \text{month}_{m} + \alpha_i + \epsilon_{i,m}  \\
  \label{eq:month_sd}
 y_{i,m} &= \sum_{m=2}^{12} \delta_{m3} \text{month}_{m} + \beta X_i +\epsilon_{i,m}  
\end{align}
Next, we estimate a second set of the same equations,
Equations \ref{eq:month_t}, \ref{eq:month_fe_t}, \ref{eq:month_sd_t}, 
that incorporate a dummy variable
that indicates if the participant is a honey producer. Estimates of 
$\gamma^m$ capture the difference in the probability of food insecurity in a given month between honey producers and non-producers.
\begin{align}
\label{eq:month_t}
 y_{i,m} &= \sum_{m=2}^{12} \gamma_{m1}^{\text{ols}} T_i \text{month}_{m} + 
 \sum_{m=2}^{12} \delta_{n1} \text{month}_{m} + \theta_1 T_i 
 + \epsilon_{i,m}  \\
 \label{eq:month_fe_t}
 y_{i,m} &= \sum_{m=2}^{12} \gamma_{m2}^{\text{ols}} T_i \text{month}_{m} + 
 \sum_{m=2}^{12} \delta_{n2} \text{month}_{m} + \alpha_i + \theta_2 T_i + \epsilon_{i,m}  \\
  \label{eq:month_sd_t}
 y_{i,m} &= \sum_{m=2}^{12} \gamma_{m3}^{\text{ols}} T_i \text{month}_{m} + 
 \sum_{m=2}^{12} \delta_{n3} \text{month}_{m} + \beta X_i + \theta_3 T_i +\epsilon_{i,m}  
\end{align}
Because we are using panel data, we cluster standard errors at the individual level
to account for possible correlation among unobservables of the same
individual across months.

\begin{align}
    y_{i,m} &= \phi_{\text{ols}} I[m \in \{4,5,6\}] + \beta_{\text{ols}} T_i + \theta_{\text{ols}} T_i I[m \in \{4,5,6\}] + \epsilon_{i,t}
\end{align}

\subsection{Identification Strategy}
As participants decide whether they will produce honey,
their honey producer status $T_i$ could be correlated
with other unobserved characteristics and thus bias our estimates of 
$\gamma^m$ (the effect of honey production on food insecurity
in month $m$) in Equations
\eqref{eq:month_t}, \eqref{eq:month_fe_t}, and \eqref{eq:month_sd_t}.
For example, participants of above-average farming ability 
could select into honey production and independently experience
levels of food insecurity, both because of natural ability.

For this reason, we use an instrumental variables approach (IV) to
control for this endogeneity. Following \textcite{sellare_sustainability_2020},
we construct the share of participants in the same region 
that have adopted honey production, excluding participants themselves.
\begin{align}
\label{eq:instrument}
    \hat{T_i} &= n^{-1} \sum_{j=1, j \ne i}^n T_j 
\end{align}

Next we estimate the following versions of Equations \eqref{eq:month_t}, \eqref{eq:month_fe_t}, and \eqref{eq:month_sd_t} using $\hat{T_i}$ as an
instrument for $T_i$.
\begin{align} 
\label{eq:month_t_iv}
 y_{i,m} &= \sum_{m=2}^{12} \gamma_{m1}^{\text{iv}} \hat{T_i} \text{month}_{m} + 
 \sum_{m=1}^{12} \delta_{n1} \text{month}_{m} + \theta_4 \hat{T_i}
 + \epsilon_{i,m}  \\
 \label{eq:month_fe_t_iv}
 y_{i,m} &= \sum_{m=2}^{12} \gamma_{m2}^{\text{iv}} \hat{T_i} \text{month}_{m} + 
 \sum_{m=1}^{12} \delta_{n2} \text{month}_{m} + \alpha_i + \theta_5 \hat{T_i} + \epsilon_{i,m}  \\
  \label{eq:month_sd_t_iv}
y_{i,m} &= \sum_{m=2}^{12} \gamma_{m3}^{\text{iv}} \hat{T_i} \text{month}_{m} + 
 \sum_{m=1}^{12} \delta_{n3} \text{month}_{m} + \beta X_i + \theta_6 \hat{T_i} +\epsilon_{i,m}      
\end{align}

The credibility of our instrument depends on whether it satisfies the
relevance condition and the exclusion restriction. First we discuss
the relevance condition. As mentioned previously,
indigenous farmers in this reason who begin producing honey rely
on loans of equipment from their neighbors. Thus it makes sense
that the share of honey producers in the same region would be correlated
with an individual's honey production status. Table ? shows results
of estimating Equation \eqref{eq:instrument}.

Next, we discuss the exclusion restriction. According to the exclusion
restriction, the instrument only affects the outcome variable via the treatment.
In other words, the share of one's neighbors who are honey producers
only affects food insecurity through one's own honey production. A potential
objection to the exclusion restriction is as follows. One's neighbors could
have extra food by virtue of their own honey production. They could share
food with a given individual and thus alleviate his or her food insecurity by
a channel that is not the individual's own honey production. 

We control
for this possible alternate channel in two ways. 
First, we estimate modified versions of equations
\eqref{eq:month_t_iv}
\eqref{eq:month_fe_t_iv}, and \eqref{eq:month_sd_t_iv}, and
above with regional fixed effects. The regional
fixed effects difference out excess food in a region and 
potentially control for
this potential violation of the exclusion restriction. Second,
we use the falsfication test proposed by \textcite{di_falco_does_2011}.
If the share of adopting neighbors above is a valid instrument, it will affect 
an individual's technology
adoption decision but will not affect the outcome variable
(monthly food insecurity) for the participants who do not adopt the technology. 

Tables ? and ? show that our instrument passes both of these
tests and therefore that it is a valid instrument.

\begin{table}[p]
    \centering
    \caption{Monthly Variation in Food Insecurity}
    \include{tables/table1}
    \label{tab:table1}
\end{table}

\begin{landscape}
\begin{table}[p]
    \centering
    \caption{Honey Production and Monthly Variation in Food Insecurity}
    \include{tables/table2}
    \label{tab:table2}
\end{table}
\end{landscape}

\begin{table}[p]
    \centering
    \caption{Honey Production and Monthly Variation in Food Insecurity}
    \include{tables/table3}
    \label{tab:table3}
\end{table}


\begin{table}[p]
    \centering
    \caption{Honey Production and Monthly Variation in Food Insecurity}
    \include{tables/table4}
    \label{tab:table4}
\end{table}

\end{document}
