\documentclass[../main.tex]{subfiles}

\doublespacing

\begin{document}

This thesis examines the relationship between coffee producers who diversified into beekeeping, and their exposure to food insecurity during the honey harvest season. This was done using an instrumented difference-in-difference approach that controls for the individual characteristics of the subject, along with their exposure to food insecurity in the previous 9 months, and compares the difference in outcomes during the honey harvest season when compared to coffee producers who did not take up beekeeping. 

What this study finds is a negative relationship between honey seasonality and food insecurity; when controlling for demographic characteristics, individual fixed effects, and endogeneity of take-up, producing honey is negatively associated with food insecurity exposure during the honey harvest season. In addition, the difference in means in total food insecurity duration is lower for honey producers, which aligns with previous studies in Chiapas, Mexico (Anderzén et al., 2020).

This paper contributes to the literature on food insecurity among agricultural producers by exploring the causal mechanisms behind the relationship between honey production and food insecurity and the adoption of honey production. Previous literature has described the associations between food insecurity, honey production, and other coping strategies descriptively but not causally. The findings in this paper can inform future research and public policy to improve smallholder producers’ food security.

Future work can enhance the foundation set in this study by conducting a longitudinal study of the Tseltal community, with annual surveys on food insecurity timeline, and coffee harvest yields. Most importantly, either observing future coffee producers who take-up learning how to be a beekeeper, or by facilitating a controlled environment for who is given the incentive to start beekeeping will enhance the validity of the parallel trends assumption and provide deeper understanding of honey production affects the economic wellbeing of Tseltal coffee producers.

\end{document}
