\documentclass[../main.tex]{subfiles}

\doublespacing

\begin{document}

\subsection{Collection of primary and secondary data}

To investigate the relationship between honey production and food security, this study provides primary data on 275 coffee producers, collected between the months of June and August 2022 in the central highland region within the Mexican state of Chiapas (Figure \ref{fig:map_chiapas_map}). This region is home to the Tseltal Mayan indigenous community, where most households are involved in agricultural production and the primary source of income comes from coffee. These farms are not privately owned, but are located on community-owned ejido lands. The village centers are located in remote mountainous regions that limit accessibility of visitation and coordination. Additionally, the majority of the coffee producers in this region speak the Tseltal Mayan dialect and low Spanish fluency. To mitigate these obstacles, we partnered with the locally-based organization, Yomol A’tel, along with its coffee and honey subsidiary cooperatives: Bats’il Maya \& Chabtic,  for logistical management. Yomol A’tel has strong relationships within the Tseltal communities, and has Tseltal representation within the organization as well. 

Prior to data collection, the research team visited the Chiapas city of Chilón, to with Yomol A’tel leadership, and Tseltal community representatives. We presented our research agenda, and received approval to visit the community centers and to conduct our research. The representatives then helped in broadcasting the survey schedule to allow coffee producers within the region to make necessary accommodations to be available to participate. Yomol A'tel facilitated the hiring of 10 enumerators who were fluent in both Spanish and Tseltal, in addition to managing travel logistics.  

\subsubsection{Survey Questions}

Due to the survey involving financial questions, participants were first given a pre-survey evaluation consisting of three mathematical filter questions: (1) percentage calculation stated as “What is 40\% of 100 MXN?”, (2) simple arithmetic in context of coffee stated as “If you produce 17 bags of coffee and sell 9, how many bags do you still have?”, and a probability question “If there are 3 blue balls and 7 red balls in a bag, and you pick a ball at random. Is it more probable that it is red or blue?” This last question was visually demonstrated by the enumerator, where all 10 balls are displayed to the participant, and then placed into a small burlap bag. The enumerator then asks if they were to draw a ball from the bag, which color is more likely to be selected.  If the participant stated they were a coffee producer, and successfully answered 2 of the 3 math problems, they were permitted to take the survey.

The survey consisted of 40 questions, categorized into 5 sections: household demographics, income and food insecurity, farm characteristics, coffee production, and honey production . Household demographics asked questions regarding to the participant’s age, gender, family size, and dependents. Next participants were asked about how much farm-related income they earned in the past year, along with how much they earned from off-farm sources (e.g., working on a neighboring farm, or seasonal internal migration work). While income being present in mind, participants were then asked to select all months in the past year that there was not enough food to feed their family. Farm characteristics asked participants to select all the types of livestock and crops they have on their farm, along with the hectare size of their farm and a rank-order list of preferred change to their farm if they encountered a sudden boost in income. In the coffee production section, participants were asked how much coffee did they harvest in the past year, how much of their harvest did they sell to between the cooperative and local intermediaries (commonly referred to as “coyotes”), the highest and lowest prices they received, and what types of coffee varietals do they plant on their farm. 12 different coffee varietals were selected to choose from, half were arabica varieties that are susceptible to coffee leaf rust, while the other half consisted of arabica hybrid cultivars and Robusta varieties that are resistant to coffee leaf rust and other pests. Lastly, participants were asked if they also produce honey on their farm. If the participant stated that they were, they were then asked how much they harvested in the past year, and how much they sold their honey for. If participants stated they do not produce honey, then they were asked to rank on a scale of 1 to 10 their level of interest to start producing honey. 

\subsubsection{Data Collection}

The enumerators were trained on how to operate tablet devices, where the survey question list was constructed using the open-source platform, Kobo Toolbox.  The research team consisted of 2 doctoral students, 1 graduate student, 10 enumerators, and 3 Yomol A’tel project leaders. This study was conducted in 11 village centers within the municipalities of Chilón, Ocosingo, Sitalá, Pantelho, and Yajalón, where a total of 275 coffee producers participated in the survey (Figure \ref{fig:map_survey_regions}). Of those who participated, 128 (47\% of sample population) were coffee cooperative members, 54 (20\% of sample population) were honey producers, and 21 of the honey producers were also in the honey cooperative (55\% of cooperative members).

Yomol A’tel’s coffee cooperative, Bats’il Maya, provided extensive data on their 741 cooperative members, along with transaction receipts of coffee graded and purchased from cooperative members from the years of 2013 through 2019. Chabtic is the latest cooperative within Yomol A’tel, with 38 honey producing cooperative members. We were given data on membership and 2022 transaction receipts. In addition, prior household surveys were conducted \parencite{pitts_value_2019} was also used to match participants in this study.

\subsection{Household Survey}

Without controlling selection to participate, this study achieved balanced gender representation with 138 males, and 137 females. The average age in this study was 43.4 years (15.6 std. dev.) The widespread in age is due to non-head-of-household and head-of-household members alike participating in the survey,  we did not exclude non-head-of-household representatives from participating due to farm labor and management is shared by multiple members within the family. For education, 75\% of sample reported only receiving primary education, while only 29 (10.5\%) reported finishing high school. While the education level is low and 74\% reported to be able to read and write, 99\% of respondents passed basic math problems of calculating sales percentages and inventory subtraction. The only problem met with difficulty was a math problem based on probability, which 74\% respondents answered correctly. Family size averaged was 6.8 members (3.1 std. dev.), with 2.4 members on average being either under the age of 12, or above 65 (Table \ref{summary_stats}).

\subsubsection{Farm Characteristics}

Participants reported on average having 3.65 hectares of farmland, with 81\% stating over 70\% of their farmland’s primary use was for coffee production. Since coffee is typically grown under the shade canopy of larger trees, the land allocated to coffee does not prohibit the presence of other crops in that area. In addition to growing coffee, participants reported on average to be growing 9 of the 15 possible types of crops on their farm. 

On average, participants reported having 2.4 (1.3 std. dev.) types of coffee trees on their farm, and 34\% (.31 std. dev.) of those types were CLR-resistant hybrids (Table \ref{summary_stats}). None of the household demographics displayed any significant relationship to the percentage of CLR-resistant hybrids on a given farm, nor the size of the farm or quantity of coffee harvested.  The most commonly-listed cultivar in the survey was Garnica, which is a cross between the Caturra and Novo Mundo varietals. Garnica is known to produce quality flavors, but due to Caturra and Novo Mundo are both susceptible to CLR, Garnica is also highly susceptible (Couttolenc-Brenis et al., 2020). The second most common cultivar was Oro Azteca, which is a CLR-resistant hybrid. On the other extreme, the least common was Robusta. Regardless of the threat to a CLR outbreak, participants have shown a preference to adopting arabica hybrid cultivars instead of defaulting to the safest alternative with the low-grade Robusta variety (Table \ref{varietals}).

In the post-harvest stages of coffee production, coffee cherries need to be depulped to remove the seeds from the rest of the fruit,  then a drying stage breaks off the protective parchment shell to arrive at the exportable product of green coffee. Coffee producers can sell their coffee to an intermediary at any point in this process, but it is most common to sell at the parchment stage or the green coffee stage. On average, coffee producers in sample harvested 371kg. of green coffee on average. In the process of selling parchment and green coffee, sales are measured in quintals, which are 60kg. bags. Thus, the average harvest yielded 6 quintals (Table \ref{summary_stats}). 

\subsubsection{Honey Producers vs. Non-Honey Producers}

When comparing honey producers to non-honey producers, there were no significant differences in mean values for age, farm acreage, education level, and income. However, honey producers were found to be 9\% less likely to be female, 1 additional family member on average, and almost twice as likely to be a cooperative member over non-honey producers. In alignment with literature regarding biodiversity, honey producers had 12\% more coffee varietals, and slightly more types of food crops with 3\% higher average value. We also find higher yield and yield density on honey-producing farms with a full additional quintal’s worth of coffee harvested over non-honey producers. For total duration for food insecurity, honey producers reported 10\% less duration with 1.67 months of food insecurity (Table \ref{summary_region}). We find honey producers within 8 of the 11 survey regions, and the majority of them are concentrated within 5 regions (Figure \ref{fig:map_honey_percent}); in these regions, at least 20\% of coffee producers produce honey as well. 

\subsection{Seasonality Timeline}

Using Yomol A’tel internal records from the years 2013 – 2019, parchment coffee was sold between the months of December through April.  Meanwhile, Chabtic sales data from 2021 – 2022 found honey was collected during the months of March through June.  There is overlap between honey harvest and coffee harvest timeline, and therefore we use the months where honey sale concentrations were higher than coffee sales as our months of interest to define as the honey season. The months that satisfy this criterion are April, May, and June, and these three months account for 83\% of all honey sales (Figure \ref{fig:monthly_occurrences}).

\subsubsection{Food Insecurity Timeline}

Food insecurity was found in all 12 months within the survey, 94\% of all reported food insecurity occurred during the months of April through September.  The most frequently reported month of experiencing food insecurity was July with 47\% of sample population experienced food insecurity in that month, followed by the months of August and June where 121 and 88 of sample population reporting to have experienced food insecurity respectively (Figure \ref{fig:food_insecurity}). These high rates of reported food insecurity confirm that food insecurity is not an isolated issue but rather is an ongoing risk that the whole community is exposed to. While 49 individuals reported no months of food insecurity in the past year, 22 reported going least a third of the year without adequate food supplies. To define the period of highest risk of experiencing food insecurity, or what is commonly referred to as the “Lean Season”, we set a continuous 25\% incidence rate cutoff, which captures the successive months between June and August as the only months that more than 25\% of sample population experienced food insecurity. As for the low occurrence of reported food insecurity during the months of can be explained as preorder sales of coffee to intermediary processors and cooperatives \parencite{macchiavello_competition_2021}.

\end{document}