\documentclass[../main.tex]{subfiles}

\doublespacing

\begin{document}

Among smallholder agricultural farms, coffee producers face a combination of risk factors that leaves them particularly vulnerable to food insecurity. Green coffee price volatility (price risk), along with climate and crop disease factors (harvest risk) affect the capacity of earning enough from the sale of their coffee harvest to meet their annual food consumption needs. In addition, most coffee harvests occur once per year, which the lack of an income-smoothing mechanism adds further budgeting constraints. 

Food security among smallholder agricultural farms in low- and middle-income countries has unsurprisingly been a topic of interest within agricultural and development economics research, with much of the emphasis on mitigating quantity risk \parencite{karlan_agricultural_2014, mcintosh_utility_2019} and price risk \parencite{bellemare_welfare_2013, boyd_why_2022}. This research focuses on the fact that coffee producers are paid only once per year, at harvest time. A coffee producer may still experience months of food insecurity even in the absence of a crop or price shock, due to insufficient slack in the household budget to withstand unexpected expenses or budget management failure due to overexerted willpower \parencite{mani_poverty_2013}. For smallholder producers that are just above the minimum requirement to meet their needs, the lack of an income smoothing mechanism makes maintaining food security an incredibly difficult task. To mitigate their exposure, coffee producers can diversify their income through on-farm and off-farm alternatives. On-farm diversification can be in the form of planting more food crops, or additional cash crops, while off-farm diversification options are to work on a neighbor's farm or to internally migrate \parencite{bacon_explaining_2014}. 

To further investigate the issue of food insecurity exposure for coffee producers, this study uses primary data collected from the Mexican state of Chiapas. Mexico is one of the 10 largest coffee exporters in the world, and the state of Chiapas is Mexico’s primary coffee-producing state: accounting for 40\% of all Mexican coffee exports \parencite{siap_servicio_2022}. Coffee is prominent in Chiapas due its high elevation regions with the Sierra Madre mountain range and the Central Highlands, but also because it is Mexico’s southernmost state and borders Guatemala, another top 10 global exporter of coffee. However, Chiapas is also notable for another important statistic, it is also Mexico’s poorest state, with 75.5\% living in poverty, and 29\% living in extreme poverty \parencite{coneval_consejo_2020}. For indigenous communities where coffee is the primary source of income, these percentages are considerably higher, as over 90\% of indigenous population in Chiapas live below the poverty line. (CONEVAL, 2019). Within Chiapas, the largest indigenous population are the Mayan Tseltal community,  where the majority of live on agricultural communal lands called “ejidos”, that was returned to indigenous possession through the passing of the 1917 Mexican Constitution. For those who live in the mountainous regions, coffee is one of the primary sources of income, as most coffee farms are able to sit under the shade of the forest canopy. But in recent years a new risk has creeped into the lives of coffee producers with a deadly fungus known as Coffee Leaf Rust (CLR)  has taking turns causing regional epidemics throughout various parts of the country ever since 2012 when coffee production dropped by 46\% for the year (McCook, 2019). Food security is already a continuous concern, but in recent years the need for alternative income sources has significantly increased. 

A subset of coffee producers are also beekeepers and harvest honey on their farm. Due to honey being a popular medicinal product in many cultures and being a pollinator of coffee trees, beekeeping has long been present on coffee farms. However, recent studies have found a relationship between honey production and lowered exposure to food insecurity \parencite{anderzen_effects_2020}. One of the possible reasons for this relationship that this research focuses on is the time in which honey is harvested. While bees may produce honey throughout most of the year in coffee-growing regions, there is generally one season where there is a surplus of honey for the coffee producer to extract, and generally occurs a couple of months after the annual coffee harvest \parencite{martello_use_2022}. This occurrence is not coincidental, but rather there is link that draws coffee and honey together that takes place shortly after a coffee harvest has removed all the coffee cherries from the tree: the coffee flowering stage. While coffea arabica can self-pollinate, insect pollinators play an active role in fruit development on coffee trees, and can increase total coffee yields \parencite{hipolito_landscape_2018, khan_honey_2007}. When honeybees are located near a coffee farm, the most abundant local flowering plant will be coffee, and with a concentrated flowering stage results in a concentrated period of honey production. The coffee harvest season takes place over multiple months in a given country, but the farm-specific harvest timeline is typically within a single month. Since honey production is linked to and follows the coffee harvest timeline, this also implies that honey harvest timeline will be farm specific as well. Therefore, I hypothesize that one of the primary benefits that honey provides is a source of income that is set to arrive when food insecurity exposure is rising. 

This study explores plausible exogenous identification in the relationship between honey production and food insecurity, along with other coping mechanisms by focusing on two primary questions: (1) What is the relationship between time of year and food insecurity? (2) Does diversification into honey affect food insecurity exposure in a specific season?

During the summer of 2022, I collaborated with researchers from the University of Minnesota (Rev. Stephen Pitts \& Chris M Boyd Ph.D.) and surveyed coffee producers from the Tseltal community Figures \ref{fig:map_chiapas_map}-\ref{fig:map_ejidos}, with logistical assistance provided by the organization Yomol A’tel. I find that of those surveyed, coffee producers who diversified into honey on their farm were less likely to experience food insecurity during the honey harvest season, and lower duration of months with food insecurity overall relative to those who were not honey producers.

\end{document}
