\begin{titlepage}

\title{Sweet and Timely Income: \\
The Effect of Honey Production in Reducing Food Insecurity of Coffee Producers
\thanks{We thank Marc Bellemare, Jason Kerwin, Michael Boland, 
Bruce Wydick, and 
participants in the invited poster session of the AAEA 2023 Annual Meetings in
Washington DC. We are grateful for funding from the 
Center for International
Food and Agriculture Policy and the Minnesota
Agricultural Experiment Station in the Department of Applied
Economics at the University of Minnesota. We obtained
IRB approval through the University of Minnesota 
(IRB ID STUDY00016085). All errors are our own.}}
\author{Grant X. Storer\thanks{Research Analysist, Inter American Development Bank. email: gxstorer@usfca.edu. Address: 1300 New York
Avenue, N.W. Washington, D.C. 20577. Phone: (+1) 202-942-8211}, Stephen Pitts\thanks{Ph.D. Candidate, Department of Applied Economics, University of Minnesota. Corresponding author. Email: pitts071@umn.edu. Address: 1994 Buford Ave, St. Paul, MN 55108. Phone: (+1) 612-625-1222.}, \\ Jesse Anttila-Hughes\thanks{Associate Professor, Department of Economics, University of San Francisco. Email: jkanttilahughes@usfca.edu. Address: 2130 Fulton St., San Francisco, CA 94117. Phone: (+1) 415-422-2765.}, and
Chris M. Boyd\thanks{Assistant Professor, Department of Economics, Towson University. Email: cboydleon@towson.edu. Address: 8000 York Road, Stephens Hall 101-N, Towson, MD 21252. Phone: (+1) 410-704-2959.}}  

\maketitle

\begin{abstract}
Smallholder cash crop farmers often struggle to smooth income and consumption throughout the agricultural year, leading to potentially serious food insecurity risk. This study examines the effect of a specific type of adaptive response to this risk, honey production, using a unique sample of indigenous coffee producers in Chiapas, Mexico. We find that honey production is associated with generally reduced food insecurity during the key honey production season, when food insecurity is otherwise generally rising. Leveraging  clustering in the spatial distribution of honey production to instrument for adoption via social learning effects, we find that honey production causes a decrease in food insecurity more than large enough to offset the non-honey producing population's worsening outcomes. Our results provide insight into the efficacy of a key alternative livelihood strategies for vulnerable smallholder producers. 


\end{abstract}

\vspace{1cm} 
\textbf{\small{}JEL Codes:}{\small{} D1, D6, D8, I3, O1, Q1.}{\small\par}

\textbf{\small{}Keywords:}{\small{} Coffee, Food Insecurity, Diversification, Honey, Beekeeping, Technology Adoption, Indigenous Communities, Chiapas, Mexico.}{\small\par}

\thispagestyle{empty} 

\end{titlepage}