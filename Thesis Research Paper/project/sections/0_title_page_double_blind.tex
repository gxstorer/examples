\begin{titlepage}

\title{Sweet and Timely Income: \\
The Effect of Honey Production in Reducing Food Insecurity of Coffee Producers}


\maketitle

\begin{abstract}
Smallholder cash crop farmers often struggle to smooth income and consumption throughout the agricultural year, leading to potentially serious food insecurity risk. This study examines the effect of a specific type of adaptive response to this risk, honey production, using a unique sample of indigenous coffee producers in Chiapas, Mexico. We find that honey production is associated with generally reduced food insecurity during the key honey production season, when food insecurity is otherwise generally rising. Leveraging  clustering in the spatial distribution of honey production to instrument for adoption via social learning effects, we find that honey production causes a decrease in food insecurity more than large enough to offset the non-honey producing population's worsening outcomes. Our results provide insight into the efficacy of a key alternative livelihood strategies for vulnerable smallholder producers. 


\end{abstract}

\vspace{1cm} 
\textbf{\small{}JEL Codes:}{\small{} D1, D6, D8, I3, O1, Q1.}{\small\par}

\textbf{\small{}Keywords:}{\small{} Coffee, Food Insecurity, Diversification, Honey, Beekeeping, Technology Adoption, Indigenous Communities, Chiapas, Mexico.}{\small\par}

\thispagestyle{empty} 

\end{titlepage}