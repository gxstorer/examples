\documentclass[../main.tex]{subfiles}

\doublespacing

\begin{document}

Smallholder farmers often struggle to smooth income and consumption throughout the agricultural year \parencite{morduchIncomeSmoothingConsumption1995}, leading to potentially large welfare losses in the form of food insecurity, reduced health, and general poverty \parencite{banerjeeEconomicLivesPoor2007}. This risk is exacerbated for cash crop farmers, who face additional risks above and beyond the normal production uncertainty stemming from disease and climate in the form of output price risk from commodity markets and its volatile effects on local food prices \parencite{boydMicroeconomicsAgriculturalPrice2020}. Farmers may employ a variety of methods to attenuate this risk, including adopting novel agricultural technologies \parencite{federAdoptionAgriculturalInnovations1985}, forming farmers' cooperatives \parencite{bizikovaScopingReviewContributions2020}, and diversifying production to include additional sources of income or nutrition available at other times of year \parencite{baconBeanAnalyzingDiversified2023}. 

This paper analyzes the efficacy of consumption smoothing efforts by cash crop farmers in a novel context, focusing on a group of 275 indigenous coffee producers in Chiapas, Mexico, some of whom produce and sell honey in addition to coffee. The sale and production of honey is a classic form of risk smoothing employed by farmers \parencite{anderzenEffectsOnfarmDiversification2020}, albeit one that, like coffee itself, is not indigenous to the Americas \parencite{goulsonEffectsIntroducedBees2003}.  Food insecurity in this context ranges widely throughout the year as a function of distance from harvest time, from nearly 0\% of families reporting insufficient food in October to nearly half (47\%) saying so in July. Honey production, which peaks several months later in the year than the coffee harvest, thus has the potential to provide meaningful diversification to food-insecure households.

Identifying the effect of honey production on food insecurity and separating it from the effect of other strategies, such as working off-farm or migrating, is difficult for at least two reasons. The first is the potential for idiosyncratic farmer characteristics unrelated to their livelihood strategy to influence food security. We use producer-level fixed effects to address this possibility, effectively comparing farmers against themselves at different times of the agricultural calendar. The second is potential endogeneity of the adoption of honey production and our outcome of interest,  food insecurity throughout the calendar year, especially during honey production. Producer characteristics and strategies could plausibly coevolve with these variables, hampering identification. In order to address this concern, we draw from the technology adoption literature \parencite{fosterMicroeconomicsTechnologyAdoption2010}, observing that regional variation in the number of honey producers suggests a key role of social learning in the adoption of honey production \parencite{fosterLearningDoingLearning1995}. Considering honey production as an introduced, non-native technology \parencite{conleyLearningNewTechnology2010}, we find an association between the share of a producer's neighbors who produce honey and the probability that an individual producer will produce honey. A 10\% increase in the proportion of the producer's neighbors producing honey is associated with a 9\% increase in the likelihood that that producer will produce honey as well. We use this share as an instrument for honey production \parencite{sellareSustainabilityStandardsBenefit2020}, as it clearly satisfies the relevance condition and may plausibly satisfy the exclusion restriction. Producers cannot move easily in this indigenous context, as land parcels are typically passed down through generations. Moreover, we do not find systematic differences between regions with higher and lower shares of honey producers. We thus argue that the honey production decisions of a producer's neighbors should primarily affect a producer's food security through the channel of their own adoption of honey production, as well as any possible  increases in general productivity due to increased pollination that should be absorbed by producer fixed effects.

Our results show that honey producers are 7\% less likely to report food insecurity during honey production months compared to non-honey producers, who experience a 9\% increase in food insecurity, though the difference is not significant. Using the share of neighbors who produce honey as an instrument, we find that a 10\% increase in the share of honey-producing
neighbors is associated with a decrease of 1.9\% in food insecurity, relative to an increase in food insecurity of 1.2\%. 

This paper contributes to three different strands of the literature. First, it extends the food security literature by considering not only the number of months but also which months in the past year the producers went without food, employing a month-producer panel with twelve observations per producer instead of a cross-sectional data set with one observation per producer \parencite{barrettMeasuringFoodInsecurity2010}. This added precision gives us insight into the temporal dimension of income and food insecurity, an in particular the key role of adaptation during lean seasons. Second, it examines honey production in terms of social learning and technology adoption, taking advantage of the isolated, indigenous context of the study site to both gain leverage in identifying our main effect and to contribute to the literature on agricultural technology adoption in indigenous communities in the Americas. Lastly, this paper contributes to the diversification literature by providing rigorous evidence of the effect of a particular and common livelihood strategy with obvious complements to cash crop production. Together these results suggest that expanding honey production may serve as a valuable addition to existing livelihood strategies among coffee and other cash crop farmers. 
 
The paper proceeds as follows. Section 2 explains the context and describes the data. Section 3 details the empirical strategy: the producer-level regressions, the panel regressions, and the instrumented versions of both regressions. Section 4 presents results from these regressions.
Section 5 concludes.

\end{document}
