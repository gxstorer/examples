\documentclass[../main.tex]{subfiles}

\doublespacing

\begin{document}

This paper examines the effect of honey production as a livelihood diversification strategy for indigenous coffee producers in Chiapas, Mexico. 
Our month-producer panel  allows us to estimate not only the overall effect of honey production on food insecurity but also  the temporal dimension of food insecurity. Our results support existing
studies of the association between honey production and increased food security, and more broadly of the value of introducing diversified sources of agricultural income into cash crop production.  A clear policy implication of our work is the importance of alternative livelihood strategies in general and beekeeping in particular for coffee producers in this region. NGOs and government organizations who promote these strategies should keep in mind the importance of social learning and peer effects. 

Future work could address limitations of our study. First, we only consider the region that producers live in as a source of social learning about honey production. We do not ask them exactly how or from whom they learned to produce honey, and may as a result our instrumental variable estimates be vulnerable to a variety of homophily and contagion biases \parencite{shaliziHomophilyContagionAre2011}.
Second, our survey only captures producers' honey production and food insecurity at one point in time.  Repeat annual visits would allow us to construct a richer panel and dig deeper into producers' ongoing experience with honey production, its evolution, as well as the source and nature of their food insecurity.

\end{document}
