\documentclass[../main.tex]{subfiles}

\doublespacing

\begin{document}

Among smallholder agricultural farms, coffee producers face a combination of risk factors that leaves them particularly vulnerable to food insecurity. As commonly found in studies of food security among smallholder agricultural farms in low- and middle-income countries, coffee has production risks that consist of quantity risk (climate and disease factors) \parencite{karlan_agricultural_2014, mcintosh_utility_2019} and output price risk (volatility in the market price of coffee) \parencite{bellemare_welfare_2013, boyd_why_2022}. Both affect the capacity of earning enough from the sale of their coffee harvest to meet their annual food consumption needs. Another factor, however, also threatens their ability to feed themselves for the entire year: the timing of the payment they earn from the sale. The size limits of farms along with limited resources leads producers to cultivate coffee as a cash crop and use the proceeds from the sale of their harvest to finance all other household consumption. Coffee harvest seasons, however, are very narrow, and in most coffee-producing countries there is only one harvest per year. The sale of the harvest typically comes as an annual lump sum, and in many cases, this lump sum does last the whole year. As a result, producer households face budget shortfalls in the later months \parencite{mani_poverty_2013}.

Producer households utilize a variety of strategies to smooth consumption. These strategies fall into two categories: off-farm and on-farm income diversification methods. Off-farm, household members find short-term work locally or internally migrate \parencite{bacon_explaining_2014}. On-farm, households diversify production into other crops. One on-farm diversification method that we are interested in here is the utilization of honey bees to pollinate crops and sell the honey that they produce as an additional income source. This paper addresses the effectiveness of honey production as a coping strategy for food insecurity.

To further investigate the issue of food insecurity exposure for coffee producers, this study uses primary data collected from the Mexican state of Chiapas. Mexico is one of the 10 largest coffee exporters in the world, and the state of Chiapas is Mexico’s primary coffee-producing state: accounting for 40\% of all Mexican coffee exports \parencite{siap_servicio_2022}. Coffee is prominent in Chiapas due its high elevation regions with the Sierra Madre mountain range and the Central Highlands, but also because it is Mexico’s southernmost state and borders Guatemala, another top 10 global exporter of coffee. However, Chiapas is also notable for another important statistic, it is also Mexico’s poorest state, with 75.5\% living in poverty, and 29\% living in extreme poverty \parencite{coneval_consejo_2020}. For indigenous communities where coffee is the primary source of income, these percentages are considerably higher, as over 90\% of indigenous population in Chiapas live below the poverty line. (CONEVAL, 2019). Within Chiapas, the largest indigenous population are the Mayan Tseltal community,  where the majority of live on agricultural communal lands called “ejidos”, that was returned to indigenous possession through the passing of the 1917 Mexican Constitution. For those who live in the mountainous regions, coffee is one of the primary sources of income, as most coffee farms are able to sit under the shade of the forest canopy. But in recent years a new risk has creeped into the lives of coffee producers with a deadly fungus known as Coffee Leaf Rust (CLR)  has taking turns causing regional epidemics throughout various parts of the country ever since 2012 when coffee production dropped by 46\% for the year (McCook, 2019). Food security is already a continuous concern, but in recent years the need for alternative income sources has significantly increased. 

A subset of coffee producers are also beekeepers and harvest honey on their farm. Due to honey being a popular medicinal product in many cultures and being a pollinator of coffee trees, beekeeping has long been present on coffee farms. However, recent studies have found a relationship between honey production and lowered exposure to food insecurity \parencite{anderzen_effects_2020}. One of the possible reasons for this relationship that this research focuses on is the time in which honey is harvested. While bees may produce honey throughout most of the year in coffee-growing regions, there is generally one season where there is a surplus of honey for the coffee producer to extract, and generally occurs a couple of months after the annual coffee harvest \parencite{martello_use_2022}. This occurrence is not coincidental, but rather there is link that draws coffee and honey together that takes place shortly after a coffee harvest has removed all the coffee cherries from the tree: the coffee flowering stage. While coffea arabica can self-pollinate, insect pollinators play an active role in fruit development on coffee trees, and can increase total coffee yields \parencite{hipolito_landscape_2018, khan_honey_2007}. When honeybees are located near a coffee farm, the most abundant local flowering plant will be coffee, and with a concentrated flowering stage results in a concentrated period of honey production. The coffee harvest season takes place over multiple months in a given country, but the farm-specific harvest timeline is typically within a single month. Since honey production is linked to and follows the coffee harvest timeline, this also implies that honey harvest timeline will be farm specific as well. Therefore, the timeliness of honey sales may provide income at period of time when food insecurity exposure is rising. 

This study explores plausible exogenous identification in the relationship between honey production and food insecurity, along with other coping mechanisms by focusing on two primary questions: (1) What is the relationship between time of year and food insecurity? (2) Does diversification into honey affect food insecurity exposure in a specific season?

During the months of June - August 2022, we collected primary survey data from 275 coffee producers within the Tseltal community (Figures \ref{fig:map_chiapas_map}-\ref{fig:map_ejidos}), with logistical assistance provided by the organization Yomol A’tel. We find that of those surveyed, coffee producers who diversified into honey on their farm were less likely to experience food insecurity during the honey harvest season, and lower duration of months with food insecurity overall relative to those who were not honey producers.

The remainder of this paper is organized as follows: Section (2) explains the theoretical and empirical context regarding smallholder coffee producer livelihoods, and food insecurity coping mechanisms. Section (3) details the research design and descriptive data of sample population. The empirical design used in this study is detailed in Section (4). Section (5) presents the results of this study and is summarized in Section (6). Additional results listed in the Appendix of Section (7).

\end{document}
