\documentclass[../main.tex]{subfiles}

\doublespacing

\begin{document}

\subsection{Agriculture in Less-Developed Economies}

Coffea Arabica, the most common plant used to produce coffee cherries, grows in high altitude mountainous regions between the Tropic of Cancer and the Tropic of Capricorn, where a narrow temperature range of 59 to 77 degrees Fahrenheit for optimal production \parencite{thurston_coffee_2013}. The size and sensitivity to temperature variance has caused most coffee trees to be planted under a canopy of larger trees for much of coffee agricultural history. Furthermore, the criteria for agricultural land suitable for coffee production being mountainous forested land between the tropics means that most coffee farms rest on hillsides and not flat industrialized agricultural plots results in over 60\% of all coffee produced comes from a farm size that is less than 5 hectares \parencite{siles_smallholder_2022}.  

The simplified process to go from cherry to the consumable product is the harvest, separating the seed from the fruit processing (green coffee), and roasting. Since most coffee is produced on smallholder farms, the total farm harvest is well below minimum levels of most export shipment orders, and thus require some form of intermediary to combine processed coffee from multiple farms within the region to fill a single export order. These intermediaries tend to be a producer cooperative that individual farms are represented collectively, or through other local intermediaries. As a result, much of business transactions operates on an informal level, where incomplete and relational contracts play a larger role \parencite{macchiavello_competition_2021}. Coffee farms are family enterprises, and for most part every able-bodied member of the household is actively involved in the management and the decision-making process of the farm (Udry, 1996). 

In remote agricultural regions, community and social networks play an integral part in the dynamics of coffee production, and regional culture affects the institutions that emerge within \parencite{north_institutions_1991}. It is through these cultural practices, and the practices that sustain over time that influence institutions that have been mostly ignored by conventional agricultural and economic theory in the past, due to its primarily internal validity and limited visibility of external validity \parencite{corno_age_2020, jacoby_watta_2010, nunn_relationship-specificity_2007, ostrom_governing_1990}.

\subsection{Production Risks}

Risk in agricultural production is broken up into five broad categories: production, market, institutional, personal, and financial \parencite{komarek_review_2020}, but when controlling for external factors beyond the scope of the business exchange, income risk primarily can come from two main avenues: production risk and price risk. For coffee producers in the western hemisphere, one of the most prominent sources of production risk is exposure to Coffee Leaf Rust (CLR). Climate change and monoculture production have been two prominent contributing factors in escalating its prevalence as warmer temperatures are making more mountainous regions more habitable for the fungus, and the lack of wind breaking shade trees increase the travel radius of the fungus spores to infect neighboring plants \parencite{avelino_coffee_2015}. 

To see if shade-grown farms perform better against Coffee Leaf Rust, a study observed if there was a divergence in outcomes during the event of an outbreak or other production shocks \parencite{bacon_vulnerability_2017}. The study was conducted in Nicaragua, where coffee farmers suffered country-wide shocks in the years of: 1998, 2009, 2014 and Coffee Leaf Rust outbreak during the years of 2010 through 2014. They compared the outcomes of producers that were affiliated with either a Fair Trade organization that prioritized coffee sale prices, or a network of smaller organizations that emphasized diversification of agricultural crops. What they found was that average duration of seasonal hunger was lower for farms that derived at least half of their food consumption from subsistence production, and the probability of meeting that criteria lower in the members who were part of the Fair Trade organization. 

Food insecurity is the product of insufficient funds when consumption is needed. This deficit can be due to low wages, but it can also occur if income is not timely. Since coffee harvests typically only occur on an annual or semi-annual basis, it exposes producers to the risk of running out of funds before the next harvest. A 2020 study found that coffee farmers in Chiapas, Mexico that were diversified in selling coffee, maize, and honey more than halved the reported time exposed to food insecurity, from 2.8 months to 1.3 months \parencite{anderzen_effects_2020}. When observing the timeline of income received, what stood out was coffee and maize harvests followed similar timelines, but sale from honey occurred during the rainy season, just as food insecurity rates started to escalate.

\subsection{Price Risk}

Much of current literature has focused on production risk, but given the historical volatility of coffee prices, price risk is just as important of a topic \parencite{bacon_confronting_2005}. This issue is further amplified due to the overall shift in monoculture intensification in the coffee industry. Agnar Sandmo presents a model of how firms should operate in the presence of price uncertainty. Since coffee producers must make the allocation ahead of price confirmation, their investment into coffee production has the potential to result in a negative value if the final price is lower than the cost of production. Thus, utility conditional on the product of sale price and quantity, for a given probability of price. But there is another component that needs to be accounted for, which is the degree of risk tolerance that the coffee producer possesses. Since the producer can’t control price in the future, they are functionally a price-taker with regards to the future, and thus the only component they can control is quantity produced. This results in quantity produced being determined by the probability of [ (price * quantity) < (COGS) ] and the level of risk tolerance they possess \parencite{sandmo_theory_1971}. Thus, the less risk tolerant the producer is, the less investment into production they will make. Sandmo further hypothesizes that risk tolerance affects allocation in the marginal change of risk probability as well using the Arrow-Pratt risk aversion functions (Arrow, 1965; Pratt, 1964). 

However, there have been two recent studies on price risk that have challenged Sandmo’s theoretical model.  Marc Bellemare and a follow up study by Bellemare and Chris M. Boyd ran a lab-in-the-field studies of Peruvian potato farmers to observe how price risk affects production decisions \parencite{bellemare_producer_2020, boyd_why_2022}. What these studies conclude is that they break Sandmo’s model of proportional response to volatility, and do not reduce production when introduced to price risk. In fact, the response is non-monotonic as more risk resulted in higher production. Moreover, when compulsory price insurance was enacted, variance between risk-free scenarios and risk-present with insurance were negligible. 

\subsection{Income Smoothing}

Income smoothing as a function is a byproduct of inefficient credit systems. According to Jonathan Murdoch, a complete credit market creates smooth consumption, since any transitory income shocks are smoothed away by borrowing and saving. Any excess cash gets saved, and any deficit is borrowed out of. This can be tested by regressing household consumption on transitory income, and if the coefficient is close to zero, then the credit market is complete \parencite{morduch_income_1995}. But if there is not a complete credit system presently available, then something external is required to produce an income smoothing effect for the individual. This means that consumption is now being caused by transitory income, and thus we need to determine the factors determining transitory income, which one of the biggest factors will be risk tolerance. Those with low risk tolerance will opt for income sources that reduce the variance in transitory income, and thus lower overall income potential (Murdoch, 1990). 

Angus Deaton theorizes that in the presence of insufficient credit systems consumption is smoothed out by investing in buffer stocks that have the capacity to be converted into means of consumption at a later date \parencite{deaton_saving_1991}. Individuals seek to minimize variance in consumption, and while consumption is dependent on income, when it comes to time, individuals use varying levels of saving that don’t require a formal credit system in order to smooth out consumption that is not dependent on the time of income \parencite{deaton_saving_1991}. This is the conscientious intent of the individual, but it is not always properly executed. Just as income may have a degree of uncertainty, so does consumption, or more accurately expenses. Individuals, and especially individuals with narrow savings margins are prone to being insufficiently prepared for large unexpected events, like medical emergencies \parencite{mani_poverty_2013}.

\subsection{Food Insecurity}

The term “food security” is a common term used in development economics, and is quite visible in its absence, but defining its parameters is quite elusive. Amartya Sen observed how recent famines have been purely the product of institutions, not by the insufficient capacity to produce food \parencite{sen_poverty_1982}. He then posited the Entitlement Approach to food security, which is the Endowment Set (resources an individual possesses), the Entitlement Set (all possible combinations of resources that could be had given the current endowment set), and the Endowment Mapping (the relationship between the sets).  More recent frameworks have modified the terms to food security as the presence of three factors to a given individual: food availability, access to food, and the utilization of food. These factors are hierarchical in structure, since there first has to be food produced enough that is left available for the individual to consume (supply side), but also the individual needs to have the means to acquire (demand side), lastly the mechanisms have to be in place for the individual to conduct the exchange (efficiency \& capacity) \parencite{barrett_measuring_2010}. 

\subsection{Technology Adoption}

For many coffee producers, coffee is the primary source of income, and honey becomes a secondary source of income that operates on a different seasonal schedule. To better understand what drives that kind of change, we can view this through the lens of technology adoption. Honey lends itself nicely to work in technology adoption since it has a direct financial impact, since previous studies have found it difficult to scale up adoption when the observed effect is something beyond financial profitability \parencite{foster_learning_1995}.

Just because there is an alternative path to increasing income, does not mean uptake will take place. Tavneet Suri and Christopher Udry’s meta-analysis paper looks at various studies within African countries regarding technology adoption \parencite{suri_agricultural_2022}. What they conclude is that African countries deal with higher degrees of heterogeneity on critical issues that affect adoption rates, and the causal factors of adoption are themselves heterogeneous. In general, they’ve compiled a list of factors that have demonstrated strong influence on adoption rates in their studies: credit/liquidity constraints, insurance constraints, information constraints, transaction costs/infrastructure, and imperfect labor \& land markets all independently affect technology adoption.

In my data, we visited 11 different communities in the Tseltal Mayan region, of which 5 of those communities had at least a 20\% honey producer representation within our sample, and the other 6 communities were either below 10\% representation or had no honey producers at all. This spread in vary levels of adoption connection to one of the classic models of technology adoption theory with Griliches’ 1957 hybrid corn study \parencite{griliches_hybrid_1957}. Griliches noticed that while most states ended up with >90\% adoption rate of hybrid corn, the path to getting there had varying shades of progression. They seemed to follow an S-shape curve, but where they would start varied, and the earlier it started, the sharper the adoption rate ended up being. Griliches used a logistic growth curve to model the adoption change of a hybrid seed that ended up being dominant for each specific state, and starting the time period for each state based on when that hybrid reached 10\% adoption rate. What is found here is that adoption is dependent on the degree of change it can produce. States where the hybrid was highly profitable, uptake was faster. This corresponds to a cost-benefit risk analysis approach that corn farmers took. The more profitable the hybrid looked, the more were willing to take on the risk of adoption. While this study looked at whichever hybrid seed that ended up being dominant for a given state, this could be implemented in researching honey to see if different communities possess different attributes that make honey production more efficient and thus more profitable over others, and see if that is what determines which communities have adopted honey faster.

Another way to look at the heterogenous adoption paths of honey production is through the pathways of information. Using Conley \& Udry’s 2010 model of social learning’s effect on adoption rates. To be able to measure learning, the authors designed “information neighborhoods” to identify how likely a geographically close neighbor was also a source of learning to a given individual. This was done by asking the respondent if they had gone to any of 7 randomly drawn village neighbors if they went to them for farming advice \parencite{conley_learning_2010}. What they found was that farmers were more likely to change input levels of fertilizer upon receiving news that the baseline strategy was poor. But less likely to change if there’s bad news on the profitability of the alternative strategy. This insight is useful because coffee leaf rust can decimate an individual farm, thus if news comes out that a neighbor has lost a large portion of coffee trees, then uptake of diversifying in honey production may be higher. But this issue about experience with CLR also can be shown with models of technology adoption through revealed risk preferences, where a study of Chinese cotton farmers demonstrates that prospect theory does play a role in technology adoption, and lower risk tolerance was linked to low adoption rates \parencite{liu_time_2013}. This approach would need to account for which way does previous exposure to CLR affect risk tolerance. 

\subsection{Biodiversity in Coffee}

The primary focus of this study is on honey production, but that is just a subset of the sample population who are all coffee producers. With that in mind, it is important to understand the cumulative effect investment in honey production can have on coffee and the overall tropical ecosystem. As coffee became an export staple, and the influence of the Green Revolution has shifted coffee production to be focused on crop yield maximization, which has led to the transition to cutting down the tree canopy and squeezing the density of coffee tree planting as part of a monocrop plantation \parencite{hernandez-aguilera_economics_2019}. While this has helped boost crop yields, we are now becoming more aware of the hidden benefits that these old growing methods had.

Deforestation has been one of the most discussed topics within the subject of climate change, and the Amazon Rainforest, the world’s largest carbon sink, has been rapidly diminishing, along with earth’s capacity to sequester carbon dioxide from the atmosphere. In 1996, El Salvador had already lost 98\% of its original rainforest. However, 60\% of the remaining rainforest were part of coffee farms that preserved the native forest canopy \parencite{messer_can_2000}. There doesn’t always have to be a tradeoff between agriculture and forestation, and with coffee, the means of production can actively contribute to reforestation, as in the case with El Salvador, where coffee production has become a protective shield that is holding back further deforestation.

Once we understand that a coffee farm does not have to be mutually exclusive from forested land, we can begin to measure the ecological impact that a sustainable farm can have. First and foremost, one of the criteria for forestation is native tree density, and with shade-grown coffee, the process of crop production is made so that coffee trees are planted next to larger trees and interspersed amongst other crops. With this shade canopy, coffee farms could capture approximately 56 tons of above ground carbon per hectare, and over 150 tons within the soil \parencite{soto-pinto_carbon_2010}. It is estimated that if 10\% of sun-grown farms in 2010 had switched to even a low-density shade farm, over 1.5 billion tons of aboveground carbon could have been sequestered \parencite{soto-pinto_carbon_2010}. The canopy of large trees affects the soil in other ways too, in that the organic matter supplements organic nutrients, but also aids in water retention. The deep roots hold the soil together, which becomes especially important when adverse weather conditions occur. Coffee is grown along equatorial mountain ranges, which means some growing regions are vulnerable to hurricanes. When a hurricane strikes, monoculture sun-grown farms are more at risk of mudslides than shade-grown farms due to stronger tree root networks holding the soil together \parencite{philpott_biodiversity_2008}.

Biodiversity in coffee production has an ecological impact as well, since the tree canopy helps preserve native bird species to the region. This canopy is not only a destination for birds, but it also provides a link to fragmented forest regions for migratory birds, thus preserving an insular ecosystem \parencite{greenberg_bird_1997}. Birds are a natural substitute to insecticides, as birds feed on many insects that are harmful to coffee production. One of the most prevalent pests is the Coffee Berry Borer (CBB), a small beetle that eats through the coffee seeds and kills fruit development, that comes at an estimated cost of \$500 million USD in lost production annually \parencite{jaramillo_like_2011}. It just so happens that tropical birds love to feast on CBBs, and it is estimated that a single bird reduced the level of crop loss by a range of 23 to 65 lb. of a given hectare \parencite{hernandez-aguilera_economics_2019}. Birds are not the only species that migrate between forest fragments, pollinators like honey bees and butterflies are dependent on a shade canopy. Since coffee is a flowering plant, pollination is an integral part of fruit production. Thus when pollinators are preserved, it lowers the dependence on artificial fertilizers and has shown to increase yields by up to 20\% \parencite{jha_shade_2014}.

\subsection{Crop Intensification}

In coffee production we’re seeing that ever since the removal of the quota system for global coffee production, coffee intensification has been steadily increasing over time. In Latin America, 50\% of coffee farms transitioned from traditional shade-grown status to either a sparsely-shaded or complete sun-grown monoculture between the years of 1970 and 1990. This trend is further demonstrated with data collected in 2010 on 19 coffee-producing countries that concluded that 41\% of farmland was sun-grown monoculture, 24\% remained traditional shade-grown, and 35\% had a reduced sparsely-shaded polyculture \parencite{jha_shade_2014}. 

Most coffee producers are smallholders and operate on very slim profit margins. As with commodities, coffee producers are for the most part price takers, and the only factor pertaining to their income potential that they can control is production quantities. There is an increasing incentive to rip out the canopy because not only does that free up space and nutrients for more income-generating cash crops, but also coffee cherry production grows faster and more abundantly with direct sun exposure. 

In a 2019 study that evaluated the production difference between shade-grown and conventional sun-grown farms in Mexico, Colombia, and Peru. They found a weighted average coffee tree density of 4,104 per hectare for shade-grown farms. Meanwhile sun-grown farms can have a tree density of 7,500 per hectare or more. When comparing the net yield output between these two strategies, conventional sun-grown farms produced 2,972 pounds/ha, while shade-grown produced 2,286 pounds/ha. When accounting for variable costs, sun-grown coffee growers experience a net profit of \$0.286/lb. a 13.5\% increase in profit margin over shade-grown coffee \parencite{hernandez-aguilera_economics_2019}. 

\end{document}