\documentclass[../main.tex]{subfiles}

\doublespacing

\begin{document}

Smallholder coffee producers often struggle to buy enough food for the year with annual lump sum payment they receive for their coffee harvest. The main reason is timing: even if they receive their annual food budget from the sale of their coffee harvest, they cannot manage the money well to buy food throughout the year.  As a result, many diversify their livelihood strategy to add an additional source of income at another time of year. One such livelihood strategy is the sale and production of honey.

This paper examines the effect of honey production on food insecurity in the honey production season for a group of 275 indigenous coffee farmers in Chiapas, some of whom produce and sell honey in addition to coffee. We measure food insecurity using a retrospective panel that asks which months in the past year farmers did not have enough food to feed their family.  The time of year matters. Food insecurity ranges from nearly 0\% in October to 47\% in July. To compensate for food insecurity, farmers employ one or more livelihood strategies: honey production, working off-farm, or migrating internally.  We would like to isolate the effect of honey production in honey-producing months from the effect of other strategies, within honey producing months and at other times of the year. 

Identifying the effect of honey production on food insecurity poses difficulties for two reasons. The first is farmer-level shocks that affect food insecurity apart from the livelihood strategy that a farmer adopts. We use farmer fixed effects to control for these factors. The second is the endogeneity of the adoption of honey production and the outcome of interest, the probability of food insecurity in a month of honey production. Farmer-level unobservables could affect both of these variables. Thus we cannot simply compare the food insecurity reported by honey producers and non-producers to calculate the treatment effect of honey production. 

Instead, we take advantage of the spatial clustering of the adoption of honey production and construct an instrument for the probability of honey production: the share of a farmer's neighbors who also produce honey. The instrument resembles a propensity score. First, we show it satisfies the relevance condition: the higher the share of neighbors who produce honey, the higher the probability a given farmer will. Second, we argue that it satisfies the exclusion restriction: the honey production decisions of a farmer's neighbors only affect a farmer's food insecurity through the channel of the farmer's own adoption of honey production. Farmers cannot move easily because the ejido system impedes the buying and selling of parcels of land. Thus a higher share of honey producing neighbors is not a a proxy for an underlying systematic difference, such as living in a more generous community. We argue that the  distribution of farmers is random. 

Our results show that honey producers are 7\% less likely to report food insecurity in months of honey production, when food insecurity increases by 9\% overall. Using the share of neighbors who produce honey as an instrument, we find that the LATE of having honey producing neighbors is a decrease of 19\% in food insecurity, when food insecurity increases by 12\% overall. 

This paper contributes to three different strands of the literature. First, it extends the food insecurity literature by considering not only the number of months but also which months in the past year the farmers went without food. Thus it uses a month-farmer panel with twelve observations per farmer instead of a cross-sectional data set with one observation per farmer. This added precision gives us insight into the temporal dimension of income and food insecurity. Second, we examine honey production in terms of technology adoption. The technology adoption literature motivates our instrument: the number of a farmer's neighbors who produce honey. This instrument allows us to control for the endogeneity of the adoption of honey production. Third, we contribute to the diversification literature by providing causally identified evidence for the effect of a particular livelihood strategy. These results suggest the potential for expanding honey production as an alternative livelihood strategy in the region. 
 
The paper proceeds as follows. Section (2) explains the theoretical and empirical context regarding smallholder coffee producer livelihoods, and food insecurity coping mechanisms. Section (3) details the research design and descriptive data of sample population. The empirical design used in this study is detailed in Section (4). Section (5) presents the results of this study and is summarized in Section (6). Additional results listed in the Appendix of Section (7).

\end{document}
